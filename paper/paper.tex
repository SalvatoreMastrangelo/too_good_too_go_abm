\documentclass[12pt]{article}
\usepackage[margin=1in]{geometry}
\usepackage{amsmath}
\usepackage{amssymb}
\usepackage{graphicx}
\usepackage{hyperref}
\bibliographystyle{plainnat}
\bibliography{biblio}

\title{Document Title}
\author{Author Name}
\date{\today}

\begin{document}



\maketitle

\section{Literature Review}
\label{sec:literature}

This paper sits at the intersection of (i) digital food-sharing platforms (with a focus on Too Good To Go, TGTG), (ii) perishable-inventory economics and clearance/salvage channels, (iii) risk-sensitive decision-making under stochastic demand, and (iv) agent-based computational economics (ACE) and simulation-based optimisation for counterfactual analysis.

\subsection{Food-sharing platforms and Too Good To Go: adoption, impacts, and open questions}
Food-sharing platforms (FSPs) have emerged as a prominent organisational form within the broader “sharing economy” and circular-economy narratives, with the stated mission of reallocating surplus food that would otherwise be wasted. A growing empirical literature documents adoption patterns, stakeholder motivations, and perceived benefits and limitations. For example, Lagorio and Mangano~\cite{LagorioMangano2024} study TGTG from operators’ perspectives in urban contexts and report high satisfaction and potential customer-base benefits, while also noting that revenue effects may be limited and heterogeneous across operator types. Complementing this, Principato et al.~\cite{PrincipatoComisYuSecondi2025} use a mixed-methods approach to examine food service establishments’ engagement with food-sharing technologies, highlighting both internal and external barriers (e.g., implementation costs and client preferences) that shape the realised effectiveness of anti-waste initiatives.

Beyond surveys and case-based evidence, recent research has begun to formalise platform-level dynamics. Ranjbari et al.~\cite{RanjbariEtAl2024} develop a system-dynamics model for FSP diffusion and long-run waste-prevention performance, using TGTG in Italy as a reference case. This line of work is valuable for understanding adoption and macro trajectories, but it does not directly identify the microeconomic incentive effects on firm-level production decisions.

A central unresolved issue—directly motivating a micro-founded simulation approach—is whether units sold via “surprise” or “mystery” surplus channels are truly residual inventory or, instead, reflect planned production that anticipates the secondary market. Yu, Palumbo, and Secondi~\cite{YuPalumboSecondi2022} provide early empirical evidence from web-scraped TGTG offerings in Rome and explicitly raise the concern that the ability to post offers ahead of pickup can blur the boundary between leftover sales and advance selling. On the theory side, Yang and Yu~\cite{YangYu2023} model “surprise clearance” (surprise bags with quantity uncertainty) and compare it to no-clearance and transparent clearance regimes; importantly, they show that a clearance mechanism can improve profit and reduce store-side waste yet still increase production incentives and shift waste across actors, depending on demand and valuation structure. These findings underscore that the net effect of platforms like TGTG is not mechanically waste-reducing: outcomes depend on the interaction between demand uncertainty, pricing of the secondary channel, and firms’ objective functions.

\subsection{Perishable inventory, salvage channels, and waste: operational foundations}
The economics of waste in retail and food service is tightly linked to the theory of perishable inventory. Classic reviews emphasise that perishability fundamentally changes optimal replenishment/production decisions because unsold inventory either deteriorates physically or becomes economically obsolete, creating an endogenous “waste” margin~\cite{Nahmias1982}. In such environments, secondary channels (markdowns, salvage markets, clearance) are standard instruments for recovering value from otherwise-disposed units, but they also feed back into the firm’s primary-market decisions.

A substantial operations-research literature studies the joint determination of pricing and inventory policies under stochastic demand for perishable products. Chen, Pang, and Pan~\cite{ChenPangPan2014} analyse a finite-horizon perishable inventory problem in which demand depends on price plus a random shock and inventory can be disposed of; they establish structural properties (including concavity/monotonicity results) that clarify how optimal decisions respond to state variables and parameters. While such models typically treat the salvage or disposal environment as exogenous, they provide key microfoundations for modelling waste as an equilibrium outcome of price-setting and quantity decisions under uncertainty.

For TGTG specifically, the “surprise bag” format adds a distinctive element: consumers face uncertainty about the composition/quantity of the bundle at purchase time, and the firm can effectively treat the platform as a second selling channel with a different price and information structure. This aligns conceptually with recent work studying surprise clearance and the welfare/waste trade-offs induced by quantity uncertainty in the clearance product~\cite{YangYu2023}. Hence, a model that includes (i) a primary market with stochastic demand, (ii) a secondary channel with a discounted price and different informational structure, and (iii) an explicit mapping from unsold inventory to waste, can be viewed as a computational extension of established perishable-inventory primitives to a platform-mediated salvage market.

\subsection{Risk preferences and utility-based objectives under stochastic demand}
\label{subsec:riskprefs}
Because waste emerges under demand uncertainty, the firm’s response hinges on how it evaluates risky profit streams. The canonical normative framework is expected utility: a decision-maker with a strictly increasing, concave utility function prefers higher expected outcomes but dislikes risk, and comparative risk aversion can be defined and measured via the curvature of utility~\cite{VonNeumannMorgenstern1944,Pratt1964,Arrow1965}. In practice, however, many applied models employ tractable approximations or reduced-form risk penalties.

A prominent and widely used class is mean--variance (or mean--deviation) decision criteria. Markowitz~\cite{Markowitz1952} introduced mean--variance portfolio selection, formalising the idea that the attractiveness of a risky payoff can be summarised by its mean and variance. Tobin~\cite{Tobin1958} further developed mean--variance reasoning in an economic-choice context. Importantly, the relationship between mean--variance analysis and expected utility has been studied extensively; Markowitz~\cite{Markowitz2014} reviews conditions under which mean--variance efficient choices can approximate expected-utility maximisation for broad classes of concave utilities.

These mean--variance ideas have been adopted in inventory economics, especially in risk-averse newsvendor problems where a firm chooses quantity before demand is realised. Choi, Li, and Yan~\cite{ChoiLiYan2008} develop a mean--variance analysis of the newsvendor problem and characterise efficient-frontier properties under different cost structures. Wu et al.~\cite{WuEtAl2009} study a mean--variance newsvendor with stockout costs and show that risk aversion need not always imply lower ordering when shortage penalties are present—highlighting the importance of modelling the full payoff function, not merely the existence of risk sensitivity. In expected-utility formulations of the newsvendor, Eeckhoudt, Gollier, and Schlesinger~\cite{EeckhoudtGollierSchlesinger1995} examine how risk and prudence affect the optimal “newsboy” order, providing comparative statics that connect uncertainty and preferences to quantity choices. More general risk-measure approaches in multi-product contexts are provided by Choi, Ruszczyński, and Zhao~\cite{ChoiRuszczynskiZhao2011}, who treat the multi-product newsvendor as a portfolio problem under coherent risk measures and show how increased risk aversion shapes joint ordering decisions.

Taken together, this literature provides a direct basis for modelling a firm’s objective as a risk-adjusted function of profit outcomes under stochastic demand. A particularly transparent implementation—aligned with both mean--variance portfolio theory and mean--variance newsvendor models—is a criterion of the form
\begin{equation}
\label{eq:mean_sd_criterion}
U \;=\; \mathbb{E}[\pi] \;-\; \gamma\,\mathrm{SD}(\pi),
\end{equation}
where $\pi$ is realised profit and $\gamma \ge 0$ controls the strength of risk penalisation. This specification can be interpreted either as (i) a tractable reduced-form approximation to expected utility under suitable conditions~\cite{Markowitz2014}, or (ii) an operational decision criterion commonly used in risk-averse inventory models~\cite{ChoiLiYan2008,WuEtAl2009}. Crucially for TGTG-like platforms, the risk channel is economically meaningful: a secondary salvage market can reduce downside risk from unsold stock (thereby encouraging higher production), but it can also alter the distribution of profits and the marginal value of “buffer” inventory, with ambiguous implications for waste.

\subsection{Agent-based computational economics and simulation-based optimisation for counterfactual policy analysis}
\label{subsec:ace_simopt}
Many platform-incentive questions are intrinsically counterfactual: one observes equilibrium outcomes under the platform, but the no-platform world is not directly observable and may be confounded by selection and simultaneous changes in behaviour. Agent-based computational economics (ACE) provides a framework for constructing “computational laboratories” in which heterogeneous agents interact in a specified environment and macro outcomes emerge from micro rules~\cite{Tesfatsion2006Chapter}. ACE is frequently advocated as a complement to equilibrium and econometric approaches when the system features heterogeneity, bounded rationality, and strong interactions~\cite{FarmerFoley2009,Richiardi2012}. Epstein~\cite{Epstein2006} articulates the “generative” standard: one explains a macro pattern by “growing” it from explicit micro specifications. Related complexity-oriented work emphasises learning and inductive behaviour under uncertainty, which is particularly relevant when agents adapt to stochastic demand environments~\cite{Arthur1994}.

Within operations and supply-chain applications, simulation-based optimisation is increasingly used to tune decision rules in complex stochastic systems where closed-form solutions are unavailable or fragile. For perishable inventory, Xue et al.~\cite{XueEtAl2019} demonstrate simulation-based optimisation approaches for highly perishable food contexts with strong demand variability. Gioia, Felizardo, and Brandimarte~\cite{GioiaEtAl2023} provide an open-access example of simulation-based learning of ordering rules for perishable items, incorporating consumer heterogeneity via discrete choice. Liu and Nishi~\cite{LiuNishi2024} develop surrogate-assisted evolutionary optimisation for perishable inventory management, explicitly addressing the computational expense of simulation-based search. Finally, broader reviews highlight the growing integration of simulation, agent-based models, and optimisation techniques for supply chains under uncertainty~\cite{ShadkamIrannezhad2025}.

This methodological stream supports an approach in which (i) firm behaviour is specified via an objective/utility (possibly risk-adjusted), (ii) demand and substitution are simulated under alternative stochastic specifications, and (iii) production/ordering decisions are optimised within the simulation environment to evaluate comparative statics and platform counterfactuals. Importantly, the ACE perspective motivates reporting not only “average treatment effects” (platform vs no platform) but also interaction patterns: how platform effects depend on primitives such as demand volatility, relative prices, and preference dispersion, and how they shift as risk sensitivity $\gamma$ changes.

\subsection{Synthesis and gap}
Existing empirical work on TGTG and related applications documents stakeholder perceptions and usage patterns~\cite{LagorioMangano2024,PrincipatoComisYuSecondi2025}, and system-level modelling can illuminate adoption and long-run aggregate potential~\cite{RanjbariEtAl2024}. However, the microeconomic question of how a platform-mediated salvage channel affects firms’ production incentives under stochastic demand remains difficult to identify empirically and theoretically delicate because it depends on (i) the informational structure of the salvage product and (ii) the firm’s risk preferences and downside exposure~\cite{YuPalumboSecondi2022,YangYu2023}. The perishable inventory and risk-averse newsvendor literatures provide the relevant primitives and preference tools~\cite{Nahmias1982,ChenPangPan2014,ChoiLiYan2008,WuEtAl2009,EeckhoudtGollierSchlesinger1995,ChoiRuszczynskiZhao2011}, while ACE and simulation-based optimisation offer a disciplined methodology for counterfactual evaluation and interaction-rich comparative statics~\cite{Tesfatsion2006Chapter,Richiardi2012,ShadkamIrannezhad2025}. This paper contributes by integrating these strands to isolate the platform’s incentive effects on production and waste, and by using systematic parameter sweeps to map how outcomes change across demand specifications and key hyperparameters.



\end{document}